\documentclass[sigconf]{acmart}

\usepackage{booktabs} % For formal tables
\usepackage{listings}

% Copyright
%\setcopyright{none}
%\setcopyright{acmcopyright}
%\setcopyright{acmlicensed}
\setcopyright{rightsretained}
%\setcopyright{usgov}
%\setcopyright{usgovmixed}
%\setcopyright{cagov}
%\setcopyright{cagovmixed}


% DOI
\acmDOI{10.475/123_4}

% ISBN
\acmISBN{123-4567-24-567/08/06}

%Conference
\acmConference[WWW 2020]{The web conference}{April 2020}{Taipei, Taiwan}
\acmYear{2020}
\copyrightyear{2016}


\acmArticle{4}
\acmPrice{15.00}

% These commands are optional
%\acmBooktitle{Transactions of the ACM Woodstock conference}
\editor{Jennifer B. Sartor}
\editor{Theo D'Hondt}
\editor{Wolfgang De Meuter}


\begin{document}
\title{CodeGraph: A knowledge graph for Python code mined from the web}
\titlenote{Produces the permission block, and
  copyright information}
%\subtitle{Extended Abstract}
\subtitlenote{The full version of the author's guide is available as
  \texttt{acmart.pdf} document}


%\author{Ben Trovato}
%\authornote{Dr.~Trovato insisted his name be first.}
%\orcid{1234-5678-9012}
%\affiliation{%
%  \institution{Institute for Clarity in Documentation}
%  \streetaddress{P.O. Box 1212}
%  \city{Dublin}
%  \state{Ohio}
%  \postcode{43017-6221}
%}
%\email{trovato@corporation.com}


% The default list of authors is too long for headers.
%\renewcommand{\shortauthors}{B. Trovato et al.}


\begin{abstract}
Knowledge graphs have been proven to be extremely useful in powering diverse applications in search, natural language understanding, and even image classification.  CodeGraph is an attempt to build well structured knowledge graphs about program code in the hope that it can similarly revolutionize diverse applications around code such as code search, code understanding, refactoring, bug detection, and code automation.  We demonstrate how one might build such a graph by applying a set of generic techniques to Python code on the web.  We capture as nodes in the knowledge graph, calls to functions in popular modules.  The edges in the graph relate to code \textit{usage} in the wild (e.g., which other function tends to call this one, or which function tends to precede this one, as gleaned from program analysis), \textit{documentation} about the function (e.g., user documentation in code, usage documentation, or forum discussions), or \textit{program specific features} such as class hierarchies.  We use the WhyIs knowledge graph framework to make the graph easily extensible, and use its views feature to make the graph more usable for different sorts of applications.  We apply these techniques to 1.38M Python files, and associated documentation on the web to build a knowledge graph of XX nodes, and XX edges.  This knowledge graph will be made available to the larger community for use. 
\end{abstract}

%
% The code below should be generated by the tool at
% http://dl.acm.org/ccs.cfm
% Please copy and paste the code instead of the example below.
%
\begin{CCSXML}
<ccs2012>
 <concept>
  <concept_id>10010520.10010553.10010562</concept_id>
  <concept_desc>Computer systems organization~Embedded systems</concept_desc>
  <concept_significance>500</concept_significance>
 </concept>
 <concept>
  <concept_id>10010520.10010575.10010755</concept_id>
  <concept_desc>Computer systems organization~Redundancy</concept_desc>
  <concept_significance>300</concept_significance>
 </concept>
 <concept>
  <concept_id>10010520.10010553.10010554</concept_id>
  <concept_desc>Computer systems organization~Robotics</concept_desc>
  <concept_significance>100</concept_significance>
 </concept>
 <concept>
  <concept_id>10003033.10003083.10003095</concept_id>
  <concept_desc>Networks~Network reliability</concept_desc>
  <concept_significance>100</concept_significance>
 </concept>
</ccs2012>
\end{CCSXML}

\ccsdesc[500]{Computer systems organization~Embedded systems}
\ccsdesc[300]{Computer systems organization~Redundancy}
\ccsdesc{Computer systems organization~Robotics}
\ccsdesc[100]{Networks~Network reliability}


\keywords{Knowledge graphs}


\maketitle
\section{Introduction}
A number of different knowledge graphs have been constructed in recent years such as DBpedia \cite{dbpedia-swj}, Wikidata \cite{Vrandecic:2014:WFC:2661061.2629489}, Freebase \cite{Bollacker08freebase:a}, YAGO \cite{Suchanek:2007:YCS:1242572.1242667} and NELL \cite{Carlson:2010:TAN:2898607.2898816}, and these have provided significant advantages in a number of different application areas, such as search, semantic parsing, named entity disambiguation, information extraction, question answering and even image classification (see \cite{journals/tkde/WangMWG17} for instance, for a comprehensive review of the use of knowledge graph embeddings in applications).  Inspired by the value of these knowledge graphs for a variety of applications, we asked how one might build a knowledge graph in the domain of programs, rather than everyday facts.  There are a number of applications around code that could potentially benefit from such knowledge graphs, such as code search, code automation, refactoring, bug detection, and code optimization (\cite{Allamanis:2018:SML:3236632.3212695} for a review of a number of such applications).  Yet, there is no comprehensive resource such as a knowledge graph that can be leveraged in these applications as yet.  In this paper, we describe the specific challenges and solutions we came up with for building a knowledge graph for computer programs.  Specifically, we addressed three key questions outlined below.

\textit{What is a suitable representation for code in a knowledge graph?}
Our goal here is similar to what is typical of existing knowledge graphs - which is to extract the \textit{semantics} of code arifacts.  The first challenge is to determine what constitutes the `entities` of a code knowledge graph.  The semantics of code is often given by the use of library code simply because the use of libraries \textit{generalize} across user programs.  Capturing code semantics in terms of library usage has immediate benefits to many applications of code such as code search, code automation, refactoring etc. The 'entities' in our knowledge graph are therefore functions and classes of libraries that are used heavily in user code in the wild.  

How do we define the edges of a knowledge graph for code to capture program semantics?  For real world knowledge graphs, this amounts to associating textual elements that describe the `entities', along with properties defined often from semi-structured, tabular data.  For code, we extract textual elements that map natural language to code entities such as classes or functions to help define its semantics (e.g., documentation about specific functions or classes, Stack Overflow posts on web forums that describe its usage, etc).  In addition, we extract from code usage in the wild, program data and control flow, which specifies how libraries tend to be used in the wild.  We note that this method of representing programs is very powerful, and gets more at the semantics of code than relying on surface representations of code such as treating code as natural language tokens or using ASTs.  To compute this representation, we deploy program analysis techniques that are generalizable across programming languages, and we show how one might scale these techniques to millions of programs.  Of course, nothing precludes applications from using the surface representations of programs by themselves for a specific task.  However, we note that in many applications such as finding security bugs in code (\cite{DBLP:journals/corr/abs-1807-06756}), predicting variable names, method names, or types (\cite{DBLP:journals/corr/abs-1803-09544}), code summarization (\cite{DBLP:journals/corr/abs-1708-01837}), clone detection (\cite{White:2016:DLC:2970276.2970326}), predicting variable misuse, variable naming, deobfuscation, and method naming (\cite{DBLP:journals/corr/abs-1711-00740}, \cite{Bichsel:2016:SDA:2976749.2978422}, \cite{DBLP:journals/corr/abs-1808-01400}), data flow and control flow have been selectively added to surface representations of programs to gain a performance advantage.  Our approach provides this capability in a more general way, using state of the art techniques to follow object use across procedure calls, and to model heap usage. 

\textit{What are the concepts and properties in the knowledge graph?}
To formalize the semantics of types and properties in the knowledge graph, we re-used the PROV-O ontology \cite{733f89c65e4844f9aabcae1c276a5602} and schema.org \cite{Guha:2015:SES:2857274.2857276} to develop an ontology for representing code.  This step is key to understand for instance that a particular object for instance is used by another object in a function call as a specific named argument, or a particular question on stack overflow had as part of its answer a reference to a method from a specific library.  We developed an ontology to describe code and its associated artifacts such as stack overflow posts, with XXX concepts and YYY proproperties.

\textit{What is a useful interface for this knowledge graph?}
Inspired by Wikidata and Freebase, we wanted the knowledge graph to be \textit{extensible}, such that developers who mine more useful artifacts about code can feel free to publish those artifacts into the knowledge graph.  We adopted WhyIs \cite{} as our knowledge management and publishing framework because it provides three key valuable aspects to interfacing with the knowledge graph: (a) It provides views over the knowledge graph which can quickly help users of the graph get access to knowledge as designed for a specific application (e.g., say code search can extract a view over code and documentation with minimal knowledge of how the data is actually modeled across versions of code, etc)., (b) It provides inferencing capabilities over the knowledge graph, to allow a developer to plugin code to expand the graph (e.g., a recent paper analyzed code snippets in stack overflow posts to annotate a specific code snippet as a question; these could be 'added' to the entities existing in the graph), (c) It is designed to maintain provenance for each addition to the graph, which is critical if the graph can be extended by the community, and (d) It provides some basic entity resolution capabilities which allows entities mentioned in text to be linked directly.

We test these ideas by developing a comprehensive knowledge graph for 1.2 million Python programs on GitHub.  Our key contributions are as follows: (a) we describe a language independent approach to building the knowledge graph for code, although we apply it only to Python in this paper, (b) we open source a knowledge graph for Python programs, with XXX entities and YYYY edges, (b) for the popular libraries (defined by more than 1K imports of the library), we extracted documentation associated with functions and calls of the libraries, for a total of XXX functions and XXX classes across 404 libraries, (c) we extracted all stack overflow posts, and used search indexes to efficiently associate posts with specific functions, (d) we try to rigorously validate the different aspects of the knowledge graph that we built. 


\section{Granularity of knowledge graph}
Here we talk about what we want in a knowledge graph for code - what do we represent and how.  E.g., how do we define the interesting nodes of our graph - basically turtles which we define in a data driven way (imports of libraries that are 'popular').
\section{Mining usage patterns}
\section{A Running Example}
\label{sec:example}

Figure~\ref{running_example} from GitHub brings out some of the analysis  
challenges in constructing a knowledge graph for data science code.  
The illustrative code shows a simple example in which a CSV file is  
first read using the Pandas library on line~\ref{line:read}, then some  
matrix computations are performed on the data using Numpy on  
line~\ref{line:mat}, and then it is passed to {\tt ann\_show} for  
visualization at line~\ref{line:plotcall}.  After adjusting based on  
the type of the data starting at line~\ref{line:lencall}, the computed data is  
displayed using Matplotlib on line~\ref{line:plot}.  These frameworks  
are imported at lines~\ref{line:importplt},~\ref{line:importnp}, and~\ref{line:importpd}.  We want to  
capture this common usage pattern of {\tt plot} in our knowledge  
graph. 

 The first thing to note is that the code in this snippet is in three
sections spread over roughly 200 lines of the source file.  Techniques
based on source text or local structures such as ASTs or CFGs will not
be able to capture dependencies over this range.  Global structures like a
call graph can, but the call graph needs to contend with the dynamic
nature of Python, illustrated here by the assignment of the function
{\tt regress\_show} to the variable {\tt ann\_show} on
line~\ref{line:ann_show}.  Note that {\tt ann\_show} called on
line~\ref{line:plotcall} is not even a function at all, but rather a variable
assigned from the actual function.  Of the analysis frameworks that
support interprocedural analysis including first class function,
relatively few have been applied to Python.  We use WALA, which has.

This snippet also illustrates the API-intensive nature of such data
science code.  There are fives lines of user code, and these lines
make use of three APIs---{\tt Pandas}, {\tt Numpy}, {\tt
  Matplotlib}---and four functions from them---{\tt read\_csv}, {\tt
  mat}, {\tt to\_list}, {\tt plot}---with {\tt to\_list} being used
three times.  And simple models of such functions will not suffice, as
the return from {\tt to\_list} depends on what the argument is.  Such
intensive use of diverse API calls is typical, and just these three
APIs together have many thousands of functions.  Any manual modeling
will be daunting due to the sheer scope of the APIs.  There is no
formal static typing in this code to help; there is idiosyncratic
English API documentation of varying quality, but the precise
parametric semantics of functions like {\tt to\_list} is hard to
capture robustly in human- and machine-comprehensible English.
Luckily, as we show, our goals do not require knowing what the APIs
do.

But while the precise meanings of API calls do not always matter, we
do need to track the flow of objects between different calls.  Since
we want to capture that {\tt plot} is used on the result of a {\tt
  mat} call, we need to track data flow.  We really want an object
that represents whatever it may be that {\tt mat} returns.  Beyond
that, we need to follow accesses to that object, such as the read of
{\tt T} on line~\ref{line:mat}.  To do this, we introduce turtle
objects: every API call returns a fresh turtle, and accesses to
properties---such as {\tt T}--- return the same unknown object as its
container.  Thus the calls on {\tt read\_csv}, {\tt mat}, {\tt
  to\_list}, and {\tt  plot} all return new turtle objects.  On the
other hand, user code objects, such as the function {\tt
  regress\_show} are treated normally.  As we show, this mechanism
allows us to track data flow with sufficient precision without needing
to model APIs at all.


\subsection{validation of the knowledge graph}
Hold back 10\% edges of the graph, can we predict the other 90\%
\section{Mining Documentation associated with code}
This section will contain mining docstrings, associating it with turtles using inferencing?, mining RST/MD files, mining Stack overflow
\subsection{validation of the documentation?? how might we validate?}
\section{WhyIs} An introduction to what is our framework for the graph, and why we need such a framework.  What are the pieces of functionality on the graph that this enables. Visualization, demo, query access, provenance 
\section{statistics on the knowledge graph}
How many nodes and edges are in the graph, how many are connected to doc, how many are connected to usage etc


\bibliographystyle{ACM-Reference-Format}
\bibliography{paper}

\end{document}
