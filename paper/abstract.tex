\begin{abstract} 
Knowledge graphs have been proven to be extremely useful in powering diverse applications in search, natural language understanding, and even image classification.  CodeGraph is an attempt to build well structured knowledge graphs about program code in the hope that it can similarly revolutionize diverse applications around code such as code search, code understanding, refactoring, bug detection, and code automation.  We demonstrate how one might build such a graph by applying a set of generic techniques to Python code on the web.  We capture as nodes in the knowledge graph, calls to functions in popular modules.  The edges in the graph relate to code \textit{usage} in the wild (e.g., which other function tends to call this one, or which function tends to precede this one, as gleaned from program analysis), \textit{documentation} about the function (e.g., user documentation in code, usage documentation, or forum discussions), or \textit{program specific features} such as class hierarchies.  We use the WhyIs knowledge graph framework to make the graph easily extensible, and use its views feature to make the graph more usable for different sorts of applications.  We apply these techniques to 1.38M Python files, and associated documentation on the web to build a knowledge graph of XX nodes, and XX edges.  This knowledge graph will be made available to the larger community for use. 
\end{abstract}
